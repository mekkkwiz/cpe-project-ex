\chapter{\ifproject%
\ifenglish Experimentation and Results\else การทดลองและผลลัพธ์\fi
\else%
\ifenglish System Evaluation\else การประเมินระบบ\fi
\fi}

ในบทนี้จะกล่าวที่ผลลัพธ์ของโครงงานเมื่อสร้างเสร็จและฟังก์ชันการทำ หลักๆ จะมีการเทรนบอทให้มี
การตอบที่เป็นธรรมชาติ และมีความแม่นยำถูกต้องในการตอบ ตลอดจนไปถึงความพึงพอใจของผู้ใช้งาน
\section{การประเมินประสิทธิภาพ}
สามารถจำแบบออกเป็น 3 หัวข้อดังนี้
\subsection{ความถูกต้องในการตอบคำถาม (Accuracy)} 
โดยจะจัดทำ trining data set สำหรับการทดสอบและบันทึกผลโดยมีเป้าหมายคือมีความแม่ยำมากกว่า 85\% ขึ้นไป

\subsection{ความรวดเร็วในการทำงาน (Response time) }
โดยจะทำการทดลองส่ง request ไปยังตัว Dialogflow และวัดเวลาในการตอบกลับโดยใช้ JMeter ซึ่งเป็นเครื่องมือสำหรับวัด
ประสิทธิภาพของ software บันทึกผลและปรับปรุงการทำงาน

\subsection{ความพึงพอใจของผู้ใช้งาน (User Satisfaction)}
จะจัดทำแบบสอบถามสำหรับผู้ที่ใช้งานสำรวจความคิดเห็นว่าผู้ใช้งานโดยจะแบ่งออกเป็น x หัวข้อดังนี้
\begin{enumerate}
    \item มีความพึงพอใจหรือไม่
    \item พึงพอใจในส่วนใดเป็นพิเศษ
    \item ไม่พึ่งพอใจในส่วนใด
    \item อยากเพิ่มอะไรเข้าไปไหม
    \item ความอม่นยำเป็นอย่างไร 
\begin{enumerate}




