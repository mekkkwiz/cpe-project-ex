\chapter{\ifenglish Background Knowledge and Theory\else ทฤษฎีที่เกี่ยวข้อง\fi}

โครงการนี้เป็นการสร้างระบบตอบรับแบบอัตโนมัติโดยรับคำสั่งผ่านทางเสียงและตอบกลับด้วยเสียงออกมาทันที
โดยจะมีตัวกลางในการเชื่อมต่อและเป็นหัวใจหลักของโครงงานนี้ นั้นก็คือ dialogflow ซึ้งเป็น API AI
ที่จะเข้ามาช่วยในการสร้างระบบตอบรับแบบอัตโนมัติและเพื่อให้ dialogflow ที่รับ input เป็นข้อความเข้าใจ
คำสั่งที่ผู้ใช้พูดเราจำเป็นที่จะต้องมี API อีกตัวเข้ามาช่วย นั้นก็คือ Google Speech-To-Text ซึ้งเป็น API
ที่จะรับเสียงที่ผู้ใช้พูดออกมาแล้วแปลงเป็นข้อความเพื่อส่งให้ dialogflow เข้าใจและตอบกลับออก ซึ่งคำตอบของ
dialogflow ก็จะเป็นข้อความเพราะฉนั้นอีกสื่งที่จะขาดไปไม่ได้เลยคือ google Text-To-Speech เพื่อที่จะแปลง
คำตอบของ agent ให้กลายเป็นเสียงเพื่อที่ทำให้ผู้ใช้ได้ยินคำตอบของ agent โดยลายละเอียดทีของ API แต่ละตัวที่ใช้เป็นดังนี้

\section{เครื่องมือที่ใช้ในการพัฒนา}
\subsection{Dialogflow}
Dialogflow หรือ Api.ai เป็น product ที่ถูกพัฒนาขึ้นโดย Speaktoit ถูก Google ซื้อ และนำไปพัฒนาต่อยอดในปี
2016 และพึ่งเปลี่ยนชื่อมาเป็น Dialogflow โดยตัว Dialogflow ใช้เทคนิคด้าน Machine Learning และ natural language
processing (NLP) ทำให้เข้าใจการสนทนาของมนุษย์และสามารถในไปใช้ได้หลากหลายรูปแบบ และ Dialogflow ยังสามารถเชื่อมต่อกับ
Facebook, Twitter และอื่นๆได้ง่าย มันยังสามารถเชื่อมต่อกับ Google Cloud Speech-to-Text ได้ด้วย ซึ่งที่ผ่านมา
Dialogflow เป็นเครื่องมือสำคัญในการสร้างบทสนทนาที่ใช้บน Google Assistant~\cite{df-doc}

\subsection{Google Speech-To-Text}
API Google Speech-To-Text (gSTT) ของ Google เป็นอีกหนึ่งตัวเลือกที่ดีในการรวมการจดจำเสียงเข้ากับเว็ปแอปพลิเคชัน
โดยหลักการทำงานของ STT คือการรับเสียงเข้าไปประมวลผลและตอบกลับออกมาเป็นข้อความ~\cite{gstt-doc}

\subsection{Google Text-To-Speech}
เช่นเดียวกับ Google Speech-To-Text Google Text-To-Speech นั้นเป็นบริการของ Google Cloud platform
สำหรับการแปลงข้อความให้กลายเป็นเสียงพูด เพื่อตอบโต้กับผู้ใช้งานทำให้ผู้ใช้งานนั้นรู้สึกเหมือนได้คุยกับคนจริงๆ~\cite{gtts-doc}

\subsection{Google Firebase}
Firebase คือ Platform ที่รวบรวมเครื่องมือต่าง ๆ สำหรับการจัดการในส่วนของ Backend หรือ Server side
ซึ่งทำให้สามารถสร้าง Web Application ได้อย่างมีประสิทธิภาพ และยังลดเวลาและค่าใช้จ่ายของการทำ Server side
หรือการวิเคราะห์ข้อมูลให้อีกด้วย โดยมีทั้งเครื่องมือที่ฟรี และเครื่องมีที่มีค่าใช้จ่าย (สำหรับการ Scale) ซึ่งบริการหลักๆของ
Firebase ที่เราใช้จะมีดังนี้~\cite{fs-doc}

\begin{enumerate}
  \item Cloud Storage คือ บริการจัดเก็บไฟล์ข้อมูลในระบบ cloud ที่มีความปลอดภัยสูง เพื่อรองรับการดาวน์โหลดและอัพโหลดไฟล์ข้อมูลขนาดใหญ่
  \item Realtime Database คือ บริการจัดเก็บข้อมูลในรูปแบบของ JSON ที่มีการอัพเดทข้อมูลทันทีทุกครั้งที่มีการเปลี่ยนแปลง
  \item Cloud Functions คือ บริการที่ใช้ในการเขียนโปรแกรมที่รันบน Server โดยไม่ต้องมีการติดตั้งเครื่องมือเพิ่มเติม
\end{enumerate}

\subsection{Next.js}
Next.js เป็นเฟรมเวิร์กสำหรับ JavaScript ที่ใช้ในการสร้างแอปพลิเคชันเว็บไซต์ (web applications) และเว็บไซต์ (websites) ในส่วนของ
server-side rendering และ client-side rendering โดยใช้ React.js เป็นพื้นฐานในการพัฒนา ทำให้การเขียนเว็บไซต์ที่มีประสิทธิภาพสูง
และมีประสิทธิภาพด้านการโหลดเพจที่เร็วกว่าจากเว็บไซต์ทั่วไปที่ไม่ได้ใช้ server-side rendering ได้ง่ายขึ้น

ทฤษฎีที่เกี่ยวข้องกับ Next.js คือการใช้เทคนิคการแบ่งแยกโค้ด (code splitting) เพื่อลดเวลาการโหลดเว็บไซต์ โดย Next.js จะทำการแบ่งโค้ดตามหน้าเว็บไซต์แต่ละหน้า
และแบ่งโค้ดในรูปแบบของ bundle โดยจะแยกโค้ดที่จำเป็นต้องโหลดก่อน (critical code) และโค้ดที่ไม่จำเป็น (non-critical code)
โดยการแบ่งแยกโค้ดนี้ทำให้เว็บไซต์โหลดได้เร็วขึ้นและประหยัดแบนด์วิดท์ของเครือข่าย~\cite{react-doc}

\subsection{JS}
JavaScript คือ ภาษาคอมพิวเตอร์สำหรับการเขียนโปรแกรมบนระบบอินเทอร์เน็ต ที่กำลังได้รับความนิยมอย่างสูง Java JavaScript
เป็น ภาษาสคริปต์เชิงวัตถุ ที่เรียกกันว่า "สคริปต์" (script) ซึ่งในการสร้างและพัฒนาเว็บไซต์ (ใช่ร่วมกับ HTML) เพื่อให้เว็บไซต์ของ
เราดูมีการเคลื่อนไหว สามารถตอบสนองผู้ใช้งานได้มากขึ้น ซึ่งมีวิธีการทำงานในลักษณะ "แปลความและดำเนินงานไปทีละคำสั่ง"
(interpret) หรือเรียกว่า อ็อบเจ็กโอเรียลเต็ด (Object Oriented Programming) ที่มีเป้าหมายในการ ออกแบบและพัฒนา
โปรแกรมในระบบอินเทอร์เน็ต สำหรับผู้เขียนด้วยภาษา HTML สามารถทำงานข้ามแพลตฟอร์มได้ โดยทำงานร่วมกับ ภาษา HTML และภาษา
Java ได้ทั้งทางฝั่งไคลเอนต์ (Client) และ ทางฝั่งเซิร์ฟเวอร์ (Server)~\cite{js-doc}

\subsection{Express}
Express.js เป็น Web Application Framework ที่ได้รับความนิยมมากสำหรับการทำงานบน platform ของ Node.js
ซึ่งเป็น Server ตัวหนึ่ง โดยทั้ง Express.js และ Node.js ต่างก็ใช้ภาษา Javascript ในการพัฒนา ถ้าเป็น Web
Application Framework ในสมัยก่อน คนที่พัฒนาจะต้องมีความรู้มากกว่า 1 ภาษา ภาษาที่ทำงานทางฝั่ง Server อย่าง PHP
หรือ ASP และภาษาที่ทำงานทางฝั่ง Client อย่าง JavaScript เพื่อลดความยุ่งยากรวมถึงเวลาในการต้องเรียน รู้หลายๆ ภาษาทำให้เกิด
Node.js กับ Express.js เพียงแค่มีความรู้ Javascript ก็สามารถเขียนได้ทั้ง Server และ Client และเนื่องจากใช้
Javascript ในการพัฒนาจึงทำให้ตัว Express นั้นตอบสนองเร็วมาก~\cite{express-doc}

\subsection{OpenWeatherMap}
OpenWeatherMap คือผู้ให้บริการ API ที่ให้บริการดึงข้อมูลสภาพอากาศทั่วโลก ซึ่งให้ข้อมูลผ่าน API ซึ่งรวมถึงข้อมูลสภาพอากาศในปัจจุบันม
พยากรณ์อากาศ ซึ่งมีทั้งรายการในปัจจุบัน และข้อมูลสภาพอากาศในอดีตสำหรับแต่ละพื่นที่บนโลก โดยบริษัทได้จัดทำการคาดการณ์ปริมาณน้ำฝน
แบบไฮเปอร์โลคัลแบบนาทีต่อนาทีสำหรับแต่ละสถานที่ โดยอาศัยข้อมูลอากาศจากกรมอุตุนิยมวิทยาและข้อมูลจากสถานีตรวจอากาศสนามบิน,
สถานีเรดาร์ภาคพื้นดิน, ดาวเทียมตรวจสภาพอากาศ, ดาวเทียมสำรวจระยะไกล METAR และสถานีตรวจอากาศอัตโนมัติทั่วโลก~\cite{openweather-doc}

\subsection{IQAir}
IQAir เป็นเว็บไซต์ที่ให้บริการแผนที่คุณภาพอากาศและข้อมูลเกี่ยวกับคุณภาพอากาศในเมืองต่างๆ ทั่วโลก โดยใช้ Web API ในการให้บริการข้อมูลดังกล่าวให้กับผู้ใช้งาน
ทฤษฎีที่เกี่ยวข้องกับ Web API ของ IQAir ได้แก่การสร้าง RESTful API ที่ให้บริการข้อมูลผ่าน HTTP request จากผู้ใช้งาน และการใช้งาน OAuth
ในการรับรองตัวตนและการแสดงข้อมูลผ่าน API นั้น

RESTful API ของ IQAir จะเป็นแบบ stateless ซึ่งหมายความว่า API จะไม่เก็บข้อมูลสถานะของผู้ใช้งานไว้ แต่จะตอบกลับข้อมูลที่ผู้ใช้งานต้องการด้วยข้อมูลที่ถูกต้องและครบถ้วน
นอกจากนี้ API ยังมีการจัดเรียงข้อมูลในรูปแบบของ JSON ที่เป็นสากล เพื่อให้ผู้ใช้งานสามารถใช้งานได้ง่ายและรวดเร็ว

การใช้งาน OAuth จะช่วยให้ผู้ใช้งานสามารถรับรองตัวตนของตนเองก่อนเข้าใช้งาน API ดังกล่าว โดยผู้ใช้งานจะต้องลงทะเบียนและรับ API key ก่อนที่จะสามารถเรียกใช้ API ได้
และผู้ใช้งานจะต้องทำการเข้าสู่ระบบก่อนเพื่อรับการรับรองตัวตนและสามารถแสดงข้อมูลได้ผ่าน API

\subsection{cutt.ly}
Cutt.ly เป็นบริการสร้างลิงก์ย่อ URL ที่ให้ผู้ใช้งานสามารถย่อลิงก์ที่ยาวโดยอัตโนมัติ ซึ่งสามารถนำไปใช้งานในการแชร์ลิงก์ผ่านทางโซเชียลมีเดียหรือเว็บไซต์ต่างๆได้อย่างสะดวกและรวดเร็ว
ทฤษฎีที่เกี่ยวข้องกับ Cutt.ly คือการสร้างลิงก์ย่อ URL โดยใช้เทคโนโลยีของ URL shortener ซึ่งเป็นเทคโนโลยีที่มีการใช้งานกันมาก่อนที่เคยถูกใช้ในบริการอีเมล์ของ
Google และบริการโซเชียลมีเดียอื่นๆ โดยเทคโนโลยีนี้จะใช้การสร้างลิงก์ย่อ URL โดยการเข้ารหัส URL ต้นฉบับด้วยวิธีการที่กำหนดมา ซึ่งจะทำให้ URL ต้นฉบับสามารถย่อลงเป็น URL ที่สั้นลงได้
Cutt.ly ใช้การสร้างลิงก์ย่อ URL โดยใช้เทคโนโลยี URL shortener แต่ยังเพิ่มเติมฟีเจอร์อื่นๆ เช่น การติดตามจำนวนผู้เข้าชมลิงก์
การกำหนดชื่อเล่นให้กับลิงก์ การกำหนดวันหมดอายุของลิงก์ และฟีเจอร์อื่นๆ ที่ช่วยให้การใช้งานลิงก์ย่อ URL ของ Cutt.ly มีประสิทธิภาพและสะดวกสบายยิ่งขึ้น

\section{ทฤษฎีการออกแบบ UX/UI}
การออกแบบแอพพลิเคชั่นและเว็บไซต์ มีเรื่องมากมายที่ทำให้ดีไซน์เนอร์ปวดหัว เพราะมันไม่ใช่แค่ความสวยงาม แต่คือการออกแบบเพื่อตอบ
โจทย์การใช้งานของผู้ใช้ให้มากที่สุด และสิ่งที่เราจะพูดถึงในบทความนี้คือ “ลายแทง สำหรับนักออกแบบ” ก่อนลงมือพัฒนาจริง ในการออก
แบบ UI ที่ดี ต้องคำนึงถึงองค์ประกอบดังนี้
\begin{enumerate}
  \item Visibility คือ มีความชัดเจนที่จะบ่งบอกถึงเอกลักษณ์เฉพาะตัวของแอพ โดยคำนึงการใช้งาน และมี Concept ที่ชัดเจน
  \item Development ต้องคำนึงถึง ความสามารถในการปรับแต่งและข้อจำกัดของ platform เช่น การรองรับ การสร้างต้นแบบที่รวดเร็ว, มีคลังข้อมูล และมีชุดเครื่องมือที่รองรับ เพื่อที่จะสามารถต่อยอดและพัฒนาต่อไปได้
  \item Colors การเลือกใช้สีที่ดี จะช่วยให้ผู้ใช้เกิดการจดจำที่ดียิ่งขึ้นมากกว่าการใช้เพียงแค่สีขาว-ดำ อีกทั้งรวมถึงการเพิ่มประสิทธิภาพและดึงดูดผู้ใช้งานอีกด้วย นอกจากนั้นประโยชน์ของการใช้สียังรวมไปถึง
  \item Animation การที่มีอนิเมชั่นเพื่อที่จะดึงดูดให้ผู้ใช้ให้มาใช้แอพที่ video ดีกว่าการใช้แผนทีแบบเดิมๆ
\end{enumerate}

\section{\ifenglish%
    \ifcpe CPE \else ISNE \fi knowledge used, applied, or integrated in this project
  \else%
    ความรู้ตามหลักสูตรซึ่งถูกนำมาใช้หรือบูรณาการในโครงงาน
  \fi
}
\begin{enumerate}
\item ความรู้จากวิชา พื้นฐานของระบบฐานข้อมูล (261342) และวิชา ปฏิบัติการระบบฐานข้อมูล
(261343) ได้ถูกนำมาใช้ต่อยอดเพื่อศึกษาขั้นตอนวิธีการจัดการฐานข้อมูล เพื่อเก็บข้อมูลแผนที่และ video สำหรับ
นำมาใช้แสดงผลที่หน้า Web Application ของโครงงาน
\item ความรู้จากวิชา CPE Lab (261207)  เป็นวิชาที่เน้นการปฏิบัติจริงในการสร้างเว็บด้วย framework ต่างๆ โดยมุ่งเน้นให้นักศึกษาได้ฝึก
การวางแผนโครงการ การจัดการโครงการ และการใช้เทคโนโลยีที่เหมาะสมเพื่อแก้ไขปัญหาในโครงการ
ในการศึกษาการสร้างเว็บโดยใช้ framework Next.js, Express.js, Node.js, JavaScript, React และอื่นๆ นั้น
จะต้องมีความรู้พื้นฐานในการเขียนโปรแกรมเช่นการใช้งานภาษา JavaScript การใช้ Node.js และ Express.js ในการสร้าง backend
การใช้ React และ Next.js เพื่อสร้าง frontend และการใช้งาน API เพื่อเชื่อมต่อระหว่าง frontend และ backend ของโปรเจกต์
โดยการใช้ framework ต่างๆ เพื่อสร้างเว็บไซต์จะช่วยลดเวลาในการพัฒนาและเพิ่มประสิทธิภาพในการทำงานของเว็บไซต์ นอกจากนี้
การใช้เทคโนโลยีใหม่ๆ เช่น Next.js ยังช่วยให้เว็บไซต์มีประสิทธิภาพที่ดีกว่าเว็บไซต์ที่ไม่ได้ใช้ framework นี้ และการใช้งาน API
จะช่วยให้การเชื่อมต่อระหว่าง frontend และ backend มีความยืดหยุ่นและมีประสิทธิภาพในการทำงานมากขึ้น
\end{enumerate}
\section{\ifenglish%
    Extracurricular knowledge used, applied, or integrated in this project
  \else%
    ความรู้นอกหลักสูตรซึ่งถูกนำมาใช้หรือบูรณาการในโครงงาน
  \fi
}

ศึกษาวิธีการใช้สร้างและใช้งาน Dialogflow, Google TTS, Google STT , Cloud Storage 
และ Cloud Firestore ผ่านทางอินเทอร์เน็ตเพื่อสร้างระบบฝั่ง Web Server ที่จะคอยจัดการข้อมูลต่างๆ
ของแผนที่และอาจจะรวมไปถึงข้อมูลต่างๆ ข้อผู้ใช้
