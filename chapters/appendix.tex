
\chapter{การ setup firebase}

การตั้งค่า Firebase สำหรับโครงการนี้ ประกอบด้วยการตั้งค่า Firebase Realtime Database และ Firebase Storage
เพื่อทำการเก็บข้อมูลที่สำคัญของโปรแกรม

\section{สร้างโปรเจ็กต์ใน Firebase Console}
\begin{enumerate}
\item ไปที่ \url{https://console.firebase.google.com/} แล้วเข้าสู่ระบบด้วยบัญชี Google ของคุณ
\item เข้าไปที่ Firebase Console และคลิกที่ปุ่ม "Add project" เพื่อสร้างโปรเจ็กต์ใหม่
\item ตั้งชื่อโปรเจ็กต์และเลือกประเทศที่คุณอยู่
\item เปิดใช้งาน Google Analytics หากต้องการ
\item กดปุ่ม "Create project" เพื่อสร้างโปรเจ็กต์ใหม่
\end{enumerate}

\section{เปิดใช้งาน Firebase Realtime Database}
\begin{enumerate}
\item เลือก "Realtime Database" จากเมนูด้านซ้ายของ Firebase Console
\item กดปุ่ม "Create Database" และเลือกโหมด "Start in test mode"
\item บันทึกการตั้งค่า
\end{enumerate}

\section{เปิดใช้งาน Firebase Storage}
\begin{enumerate}
\item เลือก "Storage" จากเมนูด้านซ้ายของ Firebase Console
\item กดปุ่ม "Get started" เพื่อเริ่มต้นการตั้งค่า
\item เลือกโปรไฟล์ของโปรเจ็กต์
\item เลือกโฟลเดอร์เริ่มต้นเพื่อเก็บไฟล์
\item เลือก "Start upload" เพื่ออัปโหลดไฟล์
\end{enumerate}
มีตัว bot สำหรับ upload video ให้ในไฟล์ \verb+Google_Cloud_Function/localBot.py+ สามารถอ่านวิธีการใช้งานได้จาก
\url{https://github.com/mekkkwiz/Speech-Navigation-Chatbot/blob/master/Google_Cloud_Function/README.md}

\section{เพิ่มคีย์การเข้าถึงของ Firebase ในโปรเจกต์ของเรา}
\begin{enumerate}
\item ไปที่หน้า Console Firebase แล้วเลือกโปรเจกต์ที่เราสร้างไว้
\item เลือกเมนู Setting (icon รูปแฮมเบอร์เกอร์) จากด้านบนขวามือของหน้าจอ
\item คลิกที่โปรเจกต์ของเราแล้วเลือกแท็บ Service Accounts จากนั้นคลิก Generate New Private Key
\item ไฟล์ key จะถูกดาวน์โหลดลงบนเครื่องคุณ ให้เก็บไฟล์ key นี้ไว้ในโฟลเดอร์ DfFulfillment ของโปรเจกต์
\item เปลี่ยนชื่อไฟล์ key ให้เป็น serviceAccountKey.json
\end{enumerate}

\chapter{\ifenglish Manual\else คู่มือการใช้งานระบบ\fi}


\section{การติดตั้ง}
\begin{enumerate}
\item ดาวน์โหลด repository จาก \url{https://github.com/mekkkwiz/Speech-Navigation-Chatbot}
\item เปิด Command Prompt หรือ Terminal แล้วเข้าไปยังโฟลเดอร์ final-project-app ที่ดาวน์โหลดมา
\item ใช้คำสั่ง npm install เพื่อติดตั้ง dependencies ทั้งหมด
\item เข้าไปยังโฟลเดอร์ \verb+client_v2+ แล้วใช้คำสั่ง npm install เพื่อติดตั้ง dependencies ที่เกี่ยวข้อง
\item เข้าไปยังโฟลเดอร์ DfFulfillment แล้วใช้คำสั่ง npm install เพื่อติดตั้ง dependencies ที่เกี่ยวข้อง
\item สร้างไฟล์ .env และกำหนดค่าตัวแปรต่างๆดังนี้
\begin{verbatim}
GOOGLE_PROJECT_ID="your-project-id"
DIALOGFLOW_SESSION_ID="unique-session-id"
DIALOGFLOW_SESSION_LANGUAGE_CODE="th"
GOOGLE_CLIENT_EMAIL="your-client-email"
GOOGLE_PRIVATE_KEY="your-private-key"
\end{verbatim}
และดาวน์โหลด service account key จาก Google Cloud Platform แล้วบันทึกเป็นไฟล์
serviceAccountKey.json ในโฟลเดอร์ DFfullfillment ของโปรเจค
\end{enumerate}
\section{การ start server}

\begin{enumerate}
\item เปิด Terminal แล้วเข้าไปยังโฟลเดอร์ final-project-app ที่ดาวน์โหลดมา
\item ใช้คำสั่ง npm run dev เพื่อเริ่มต้นเซิร์ฟเวอร์
\item เปิด Terminal อีกหนึ่งหน้าแล้วเข้าไปยังโฟลเดอร์ DfFulfillment ใช้คำสั่ง npm run dev เพื่อเริ่มต้น Fulfillment Server
\item คุณอาจต้องใช้ ngrok เพื่อให้เซิร์ฟเวอร์ของ Fulfillment Server สามารถเข้าถึงจาก internet ได้ โดยดาวน์โหลดจาก \url{https://ngrok.com/download} แล้วใช้คำสั่ง ngrok http 3030 เพื่อเปิด ngrok
\item คัดลอก https url ที่ได้จาก ngrok เข้าไปไว้ใน fulfillment url ในเว็บ dialogflow
\end{enumerate}

\section{การใช้งาน}
\begin{enumerate}
\item เปิดโปรแกรมในเบราว์เซอร์โดยไปที่ \url{http://localhost:3000}
\item อนุญาตการเข้าถึงไมโครโฟนของคุณ
\item ใช้คำสั่งด้วยเสียงเพื่อนำทางในอาคาร สอบถามตำแหน่งของอาจารย์ หรือสอบถามเรื่องสภาพอากาศ
\item โปรแกรมจะตอบกลับด้วยเสียงผ่าน Google TTS API
\end{enumerate}

\section{วิธีการติดตั้งเพิ่มเติม}
สามารถอ่านวิธีการติดตั้งเพิ่มเติมได้ที่
\\\url{https://github.com/mekkkwiz/Speech-Navigation-Chatbot}
