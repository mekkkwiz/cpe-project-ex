\chapter{\ifenglish Introduction\else บทนำ\fi}

\section{\ifenglish Project rationale\else ที่มาของโครงงาน\fi}
โครงงานนี้เป็นโครงงานที่มุ่งเน้นการพัฒนา chatbot ที่สามารถทำงานได้ผ่านเสียงของผู้ใช้งาน
โดยเป้าหมายหลักของโครงงานคือการให้ความสะดวกสบายแก่ผู้ใช้งานในการค้นหาเส้นทางภายในอาคาร รวมถึงข้อมูลอื่นๆ
เช่น เวลา สภาพอากาศ และตำแหน่งของอาจารย์
โดย chatbot นี้จะมีการตอบกลับผู้ใช้งานด้วยเสียงที่สร้างจากการใช้ Text-to-Speech (TTS) technology
ซึ่งจะช่วยให้ผู้ใช้งานสามารถรับข้อมูลผ่านเสียงได้แทนการอ่านข้อความ เช่นเดียวกับการแสดงผลข้อมูลอย่างสะดวกสบายผ่าน
video ที่แสดงเส้นทางภายในอาคาร
โครงงานนี้มีวัตถุประสงค์เพื่อสร้างความสะดวกสบายแก่ผู้ใช้งานในการเข้าถึงข้อมูลและค้นหาเส้นทางภายในอาคาร
โดยไม่ต้องเสียเวลาในการค้นหาแผนที่หรือต้องถามคนอื่นๆ เพื่อหาข้อมูล และเป็นการใช้เทคโนโลยีที่ทันสมัยในการพัฒนา chatbot
ที่สามารถทำงานได้ผ่านเสียง ซึ่งเป็นเทคโนโลยีที่กำลังได้รับความนิยมเพิ่มขึ้นในปัจจุบัน

\section{\ifenglish Objectives\else วัตถุประสงค์ของโครงงาน\fi}
\begin{enumerate}
    \item เพิ่มประสิทธิภาพการเข้าถึงข้อมูลและความสะดวกสบายในการใช้งานอาคารสำหรับผู้ใช้งาน
    \item ให้ข้อมูลเพิ่มเติมเช่นเวลา, สภาพอากาศ, และตำแหน่งของอาจารย์
    \item ช่วยผู้ใช้งานในการค้นหาเส้นทางภายในอาคารโดยไม่ต้องเสียเวลาค้นหาแผนที่หรือถามคนอื่น
    \item ประยุกต์ใช้ Text-to-Speech (TTS) เพื่อสร้างเสียงสังเคราะห์ที่สื่อสารกับผู้ใช้งาน
\end{enumerate}

\section{\ifenglish Project scope\else ขอบเขตของโครงงาน\fi}
\begin{enumerate}
    \item ทําที่ตึก 30 ปี ในมหาวิทยาลัยเชียงใหม่
    \item ต้องเชื่อมต่อ Internet และรองรับการใช้งานผ่าน browser เท่านั้น
    \item ต้องใช้คอมพิวเตอร์ในการใช้งานเว็บไซต์
    \item รองรับการสนทนาและตอบคำถามภาษาไทยเท่านั้น
    \item ตอบคําถามได้ที่ละคําถามและในแต่ละคําถามอาจจะมีความตต่อเนื่องกัน
\end{enumerate}

\subsection{\ifenglish Hardware scope\else ขอบเขตด้านฮาร์ดแวร์\fi}
\begin{enumerate}
    \item ใช้คอมพิวเตอร์ในการเปิด Web browser
    \item ใช้จอในการแสดงผลข้อมูลและเส้นทางการเดินทาง
    \item ใช้ไมโคโฟนในการรับเสียงจากผู้ใช้งาน
    \item ใช้ลำโพงเพื่อตอบกลับผู้ใช้ด้วยเสียงพูด
\end{enumerate}

\subsection{\ifenglish Software scope\else ขอบเขตด้านซอฟต์แวร์\fi}
\begin{enumerate}
    \item OS: Windows 10 หรือสูงกว่า.
    \item Browser: Google Chrome.
\end{enumerate}

\section{\ifenglish Expected outcomes\else ประโยชน์ที่ได้รับ\fi}
เป็นโปรแกรมที่ช่วยอำนวยความสะดวกผู้ใช้ในด้านที่สามารถสอบถามเส้นทาง สถิติและสภาพอากาศ 
ได้ทันทีโดยไม่ต้องเข้าเว็บหาหรือสอบถามที่ประชาสัมพันธ์และในด้านของผู้ให้บริการการหรือทางเจ้าของอาคาร
ก็ไม่ต้องจัดคนมาคอยให้บริการในส่วนการแนะนำเส้นทางการเข้าใช้ห้องต่างๆภายในอาคารอีกด้วย

\section{\ifenglish Software technology\else เทคโนโลยีด้านซอฟต์แวร์\fi}
\begin{enumerate}
    \item React: library ที่ใช้ในการพัฒนา
    \item JS: ภาษาที่ใช้ในการพัฒนา
    \item Node.js: คือสภาพแวดล้อมการทำงานของภาษา JavaScript
    \item Google TTS: API ที่ใช้ในการแปลคำตอบไปเป็นเสียงเพื่อให้ผู้ใช้ฟัง
    \item Google STT: API ที่ใช้ในการแปลเสียงของผู้ใช้ไปเป็นตัวอักษรเพื่อส่งเข้าโปรแกรม
    \item OpenWeatherMap: API ที่ใช้ในการดึงข้อมูลสภาพอากาศ
    \item Google Dialogflow: เครื่องมือสำหรับสร้าง chatbot
    \item Google Firestore: database สำหรับเก็บข้อมูลแผนที่และข้อมูลอื่นๆ
\end{enumerate}

\section{\ifenglish Project plan\else แผนการดำเนินงาน\fi}

\begin{plan}{11}{2021}{11}{2022}
    \planitem{11}{2021}{10}{2022}{ศึกษาค้นคว้า}
    \planitem{1}{2022}{2}{2022}{เชื่อม API กับระบบ}
    \planitem{3}{2022}{3}{2022}{เตรียมแผนที่ภายในอาคาร}
    \planitem{3}{2022}{3}{2022}{ทำ video แนะนำเส้นทาง}
    \planitem{4}{2022}{4}{2022}{test ระบบครั้งที่ 1}
    \planitem{5}{2022}{5}{2022}{เตรียมข้อมูลสำหรับ train model}
    \planitem{5}{2022}{6}{2022}{train และ tune model}
    \planitem{6}{2022}{6}{2022}{test ระบบครั้งที่ 2}
    \planitem{7}{2022}{11}{2022}{นำระบบที่พัฒนาสำเร็จและผ่านการทดสอบแล้วไปทดลองใช้งาน}
\end{plan}

\section{\ifenglish Roles and responsibilities\else บทบาทและความรับผิดชอบ\fi}
โครงงานนี้ได้จัดทำขึ้นโดยผู้จัดทำเพียงคนเดียวคือ นายวิศรุต ติ๊บบุ่ง

\section{\ifenglish%
Impacts of this project on society, health, safety, legal, and cultural issues
\else%
ผลกระทบด้านสังคม สุขภาพ ความปลอดภัย กฎหมาย และวัฒนธรรม
\fi}
เมื่อโครงงานชิ้นนี้สำเร็จจะมีผลกระทบในด้าน ด้านสังคม ด้านสุขภาพ และด้านกฎหมายเป็นหลังดังนี้
\begin{enumerate}
    \item ด้านสังคม โครงงานนี้สามารถทำให้การค้นหาข้อมูลและการสื่อสารภายในอาคารเป็นไปอย่างมีประสิทธิภาพและรวดเร็วยิ่งขึ้น ทำให้ผู้ใช้งานสามารถประหยัดเวลาและทรัพยากร
    \item ด้านสุขภาพ โดยการลดการใช้งานมนุษย์และการใกล้ชิดกับผู้อื่น โครงงานนี้จะช่วยลดความเสี่ยงต่อการแพร่ระบาดของเชื้อโควิด-19 และเสริมสร้างสุขภาพของผู้ใช้งาน
    \item ด้านกฎหมาย โครงงานนี้ต้องปฏิบัติตามกฎหมายที่เกี่ยวข้อง อาทิ การป้องกันข้อมูลส่วนบุคคล การใช้เทคโนโลยี Text-to-Speech ตามข้อกำหนดที่ตั้งไว้ และการปฏิบัติตามข้อกำหนดที่เกี่ยวข้องในมหาวิทยาลัยเชียงใหม่
\end{enumerate}