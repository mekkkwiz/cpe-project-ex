\chapter{\ifenglish Introduction\else บทนำ\fi}

\section{\ifenglish Project rationale\else ที่มาของโครงงาน\fi}
อาคารหลายๆแห่งในตอนนี้นั้นบางอาคารใหญ่บางอาคารเล็กและยิ่งอาคารใหญ่ ยิ่งถ้าเลยมาครั้งแรกแล้วละก็ยิ่งทำให้หาห้องลำบาก
ไม่ว่าจะเป็นห้องเรียน ห้องประชุม หรือแม้กระทั้งห้องน้ำนั้นก็ยังหาลำบาก ด้วยเหตุนี้ก็ยังจะทำให้คนที่เข้ามาใช้หรือติดต่อ
งานต่างๆนั้นหาทางไปให้ถึงห้องที่ตัวเองต้องการนั้นยากเพราะไม่เคยมา ถึงจะมีแผนที่ภายในอาคารแต่ก็ยังอ่านยากอยู่ดี เพราะเหตุ
นี้จีงเป็นที่มาของโครงการนี้ BEVA (Building Enter with Voice Assistant) หรือในชื่อภาษาไทยก็คือ
เว็บแอพพลิเคชั่นสำหรับแนะนำการเข้าใช้อาคารด้วยคำสั่งเสียงและตอบกลับด้วยเสียง นั้นสร้างขึ้นมาเพื่ออำนวนความสะดวกให้กับ
ผู้ใช้งานอาคารในการหาห้องต่างๆ โดยไม่ต้องเสียเวลามาไล่หาห้องในแผนที่ทีละชั้น

\section{\ifenglish Objectives\else วัตถุประสงค์ของโครงงาน\fi}
\begin{enumerate}
    \item เพื่อพัฒนาระบบแนะนำการเข้าใช้ห้องต่างๆภายในอาคาร แบบ real-time
    \item เพื่อลดการใช้มนุษย์โดยให้ BEVA เข้าไปทำงานแทนในส่วนของการตอบคำถามประเภทเส้นทาง
    และรวมไปถึงถามเวลาเข้าใช้ห้องประชุมหรือห้องเรียนต่างๆในสถานการณ์ที่เชื้อ covid-19 กำลังระบาด
    \item เก็บสถิติผู้เข้าใช้อาคารเพื่อนำไปใช้ในการพัฒนาการให้บริการของอาคารนั้นๆในอนาคต
\end{enumerate}

\section{\ifenglish Project scope\else ขอบเขตของโครงงาน\fi}
\begin{enumerate}
    \item ลองทำที่ตึก ITSC
    \item ต้องเชื่อมต่อ Internet
    \item ต้องใช้คอมพิวเตอร์ในการเปิดเว็บ
    \item รองรับเฉพาะภาษาไทย
    \item ตอบคำถามได้ที่ละ 1 คำถามโดยแต่ละคำถามอาจจะมีความตต่อเนื่องกัน
\end{enumerate}

\subsection{\ifenglish Hardware scope\else ขอบเขตด้านฮาร์ดแวร์\fi}
\begin{enumerate}
    \item ใช้คอมพิวเตอร์ในการเปิด Web browser
    \item ใช้จอในการแสดงผลข้อมูลและเส้นทางการเดินทาง
    \item ใช้ไมโคโฟนในการรับสัญญาณเสียงจากผู้ใช้งาน
    \item ใช้ลำโพงเพื่อตอบกลับผู้ใช้ด้วยเสียงพูด
\end{enumerate}

\subsection{\ifenglish Software scope\else ขอบเขตด้านซอฟต์แวร์\fi}
\begin{enumerate}
    \item OS: Windows 10 or Linux.
    \item Browser: Google Chrome, Firefox or Edge.
\end{enumerate}

\section{\ifenglish Expected outcomes\else ประโยชน์ที่ได้รับ\fi}
เป็นโปรแกรมที่ช่วยอำนวยความสะดวกผู้ใช้ในด้านที่สามารถสอบถามเส้นทาง สถิติและสภาพอากาศ 
ได้ทันทีโดยไม่ต้องเข้าเว็บหาหรือสอบถามที่ประชาสัมพันธ์และในด้านของผู้ให้บริการการหรือทางเจ้าของอาคาร
ก็ไม่ต้องจัดคนมาคอยให้บริการในส่วนการแนะนำเส้นทางการเข้าใช้ห้องต่างๆภายในอาคารอีกด้วย

\section{\ifenglish Software technology\else เทคโนโลยีด้านซอฟต์แวร์\fi}
\begin{enumerate}
    \item React: library ที่ใช้ในการพัฒนา
    \item JS: ภาษาที่ใช้ในการพัฒนา
    \item Node.js: คือสภาพแวดล้อมการทำงานของภาษา JavaScript
    \item Google TTS: API ที่ใช้ในการแปลคำตอบไปเป็นเสียงเพื่อให้ผู้ใช้ฟัง
    \item Google STT: API ที่ใช้ในการแปลเสียงของผู้ใช้ไปเป็นตัวอักษรเพื่อส่งเข้าโปรแกรม
    \item OpenWeatherMap: API ที่ใช้ในการดึงข้อมูลสภาพอากาศ
    \item Google Dialogflow: เครื่องมือสำหรับสร้าง chatbot
    \item Google Firestore: database สำหรับเก็บข้อมูลแผนที่และข้อมูลอื่นๆ
\end{enumerate}

\section{\ifenglish Project plan\else แผนการดำเนินงาน\fi}

\begin{plan}{11}{2021}{11}{2022}
    \planitem{11}{2021}{10}{2022}{ศึกษาค้นคว้า}
    \planitem{1}{2022}{2}{2022}{เชื่อม API กับระบบ}
    \planitem{3}{2022}{3}{2022}{เตรียมแผนที่ภายในอาคาร}
    \planitem{3}{2022}{3}{2022}{ทำ video แนะนำเส้นทาง}
    \planitem{4}{2022}{4}{2022}{test ระบบครั้งที่ 1}
    \planitem{5}{2022}{5}{2022}{เตรียมข้อมูลสำหรับ train model}
    \planitem{5}{2022}{6}{2022}{train และ tune model}
    \planitem{6}{2022}{6}{2022}{test ระบบครั้งที่ 2}
    \planitem{7}{2022}{11}{2022}{นำระบบที่พัฒนาสำเร็จและผ่านการทดสอบแล้วไปทดลองใช้งาน}
\end{plan}

\begin{plan}{12}{2022}{12}{2023}
    \planitem{12}{2022}{2}{2023}{ประเมินผลและสำรวจความคิดเห็นของผู้ใช้}
\end{plan}

\section{\ifenglish Roles and responsibilities\else บทบาทและความรับผิดชอบ\fi}
โครงงานนี้ได้จัดทำขึ้นโดยผู้จัดทำเพียงคนเดียวคือ นาย วิศรุต ติ๊บบุ่ง

\section{\ifenglish%
Impacts of this project on society, health, safety, legal, and cultural issues
\else%
ผลกระทบด้านสังคม สุขภาพ ความปลอดภัย กฎหมาย และวัฒนธรรม
\fi}
ถ้าหากโครงการนี้สำเร็จอาจจะเกิดผลกระทบด้านความปลอดภัยและสุขภาพเป็นหลักเพราะว่า โครงการของเราสามารถเชื่อต่อกับ api ที่เกี่ยวกับ
การทำ face recognition เพื่อที่จะสามารถรับรู้ได้ว่าคนนี้คือใครและสามารถบันทึกข้อมูลของคนนั้นเก็บไว้บน cloud เพื่อช่วยในการรักษาความปลอดภัย
ของผู้เข้าใช่อาคารได้ซึ่งเป็นหนึ่งใน use caseของโครงการนี้ ส่วนทางด้านของสุขภาพ นั้นก็เป็นอีกหนึ่งเป้าหมายของเราเพื่อที่จะลดการใช้ทรัพยากรมนุษย์ 
ในสถานะการที่มีการแพร่ระบาดของ virus covid-19 เราจะนำโครงการนี้ไปติดตั้งที่อาคารต่างๆเพื่อให้บริการด้านเส้นทางภายในอาคารและลดความเสี่ยงต่อ
การเกิดอันตรายจากการที่มีการใกล้ชิดกับผู้อื่น พร้อมทั้งยังช่วยลดการแพร่ระบาดอีกด้วย