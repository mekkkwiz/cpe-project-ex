\chapter{\ifenglish Introduction\else บทนำ\fi}

\section{\ifenglish Project rationale\else ที่มาของโครงงาน\fi}
อาคาร ITSC เป็นศูนย์กลางแห่งการให้บริการเทคโนโลยีสารสนเทศของมหาวิทยาลัยและได้มีการให้บริการให้เช่าห้องประชุมและห้อง
Studio แต่อาคาร ITSC นั้นมีห้องเป็นจำนวนมากและแผนที่อ่านยาก ซึ่งจะทำให้หลายคนที่ต้องมาใช้ห้องนั้นหาห้องที่ต้องการใช้ไม่พบ

\section{\ifenglish Objectives\else วัตถุประสงค์ของโครงงาน\fi}
\begin{enumerate}
    \item เพื่อพัฒนาระบบแนะนำการเข้าใช้ห้องประชุมในตึก ITSC แบบ real-time
    \item เพื่อลดการใช้มนุษย์ในสถานการณ์ที่เชื้อ covid-19 กำลังระบาด
    \item เก็บสถิติผู้เข้าใช้อาคารเพื่อนำไปใช้ในการพัฒนาในอนาคต
\end{enumerate}

\section{\ifenglish Project scope\else ขอบเขตของโครงงาน\fi}
\begin{enumerate}
    \item ทำที่ตึก ITSC
    \item ต้องเชื่อมต่อ Internet
    \item ต้องใช้คอมพิวเตอร์ในการเปิดเว็บ
    \item รองรับเฉพาะภาษาไทย
    \item ตอบคำถามได้ที่ละ 1 คำถาม
\end{enumerate}

\subsection{\ifenglish Hardware scope\else ขอบเขตด้านฮาร์ดแวร์\fi}
\begin{enumerate}
    \item Processor: 1 gigahertz (GHz) or faster processor or SoC.
    \item RAM: 1 gigabyte (GB) for 32-bit or 2 GB for 64-bit.
    \item Hard disk space: 16 GB for Windows 10 32-bit, 20 GB for Windows 10 64-bit.
    \item Display: 800x600 or higher resolution.
    \item Microphe: Uni-directional Dynamic Microphone.
    \item Speaker: Portable Mono speaker 
\end{enumerate}

\subsection{\ifenglish Software scope\else ขอบเขตด้านซอฟต์แวร์\fi}
\begin{enumerate}
    \item OS: Windows 10 or Linux.
    \item Browser: Google Chrome, Firefox or Edge.
\end{enumerate}

\section{\ifenglish Expected outcomes\else ประโยชน์ที่ได้รับ\fi}
เป็นโปรแกรมที่ช่วยอำนวยความสะดวกผู้ใช้ในด้านที่สามารถสอบถามเส้นทาง สถิติและสภาพอากาศ 
ได้ทันทีโดยไม่ต้องเข้าเว็บหาหรือสอบถามที่ประชาสัมพันธ์และในด้านของผู้ให้บริการการหรือทางเจ้าของอาคาร
ก็ไม่ต้องจัดคนมาคอยให้บริการในส่วนการแนะนำเส้นทางการเข้าใช้ห้องต่างๆ 

\subsection{\ifenglish Software technology\else เทคโนโลยีด้านซอฟต์แวร์\fi}
\begin{enumerate}
    \item React: library ที่ใช้ในการพัฒนา
    \item JS: ภาษาที่ใช้ในการพัฒนา
    \item Google TTS: API ที่ใช้ในการแปลคำตอบไปเป็นเสียงเพื่อให้ผู้ใช้ฟัง
    \item Google STT: API ที่ใช้ในการแปลเสียงของผู้ใช้ไปเป็นตัวอักษรเพื่อส่งเข้าโปรแกรม
    \item weather API: API ที่ใช้ในการดึงข้อมูลสภาพอากาศ
    \item Google Dialogflow: เครื่องมือสำหรับสร้าง chatbot
    \item Google Firebase: database สำหรับเก็บข้อมูลแผนที่และข้อมูลอื่นๆ
\end{enumerate}

\section{\ifenglish Project plan\else แผนการดำเนินงาน\fi}

\begin{plan}{11}{2021}{11}{2022}
    \planitem{11}{2021}{10}{2022}{ศึกษาค้นคว้า}
    \planitem{1}{2022}{2}{2022}{เชื่อม API กับระบบ}
    \planitem{3}{2022}{3}{2022}{เตรียมแผนที่ภายในอาคาร}
    \planitem{3}{2022}{3}{2022}{ทำ video แนะนำเส้นทาง}
    \planitem{4}{2022}{4}{2022}{test ระบบครั้งที่ 1}
    \planitem{5}{2022}{5}{2022}{เตรียมข้อมูลสำหรับ train model}
    \planitem{5}{2022}{6}{2022}{train และ tune model}
    \planitem{6}{2022}{6}{2022}{test ระบบครั้งที่ 2}
    \planitem{7}{2022}{11}{2022}{นำระบบที่พัฒนาสำเร็จและผ่านการทดสอบแล้วไปทดลองใช้งาน}
\end{plan}

\begin{plan}{12}{2022}{12}{2023}
    \planitem{12}{2022}{2}{2023}{ประเมินผลและสำรวจความคิดเห็นของผู้ใช้}
\end{plan}

\section{\ifenglish Roles and responsibilities\else บทบาทและความรับผิดชอบ\fi}
โครงงานนี้ได้จัดทำขึ้นโดยผู้จัดทำเพียงคนเดียวคือ นาย วิศรุต ติ๊บบุ่ง

\section{\ifenglish%
Impacts of this project on society, health, safety, legal, and cultural issues
\else%
ผลกระทบด้านสังคม สุขภาพ ความปลอดภัย กฎหมาย และวัฒนธรรม
\fi}
\begin{itemize}
    \item ผลกระทบด้านสังคม:ไม่มีผลกระทบ
    \item ผลกระทบด้านความปลอดภัย:ไม่มีผลกระทบ
    \item ผลกระทบด้านสุขภาพ:ไม่มีผลกระทบ
    \item ผลกระทบด้านกฎหมาย:ไม่มีผลกระทบ
    \item ผลกระทบด้านวัฒนธรรม:ไม่มีผลกระทบ
\end{itemize}
