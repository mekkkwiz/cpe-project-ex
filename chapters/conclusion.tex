\chapter{\ifenglish Conclusions and Discussions\else บทสรุปและข้อเสนอแนะ\fi}

\section{\ifenglish Conclusions\else สรุปผล\fi}

โครงงานนี้เป็นโครงการที่ใช้การพัฒนาแชทบอทผ่าน Dialogflow โดยเชื่อมต่อกับ Web application ด้วย NodeJS และใช้ Firebase
ในการจัดเก็บข้อมูลและสื่อสารกับ Dialogflow โดยโครงงานนี้สามารถตอบคำถามได้ถูกต้องมากกว่า 85\% และมีความแม่นยำอยู่ที่ 86.66\%
โดยเวลาในการตอบคำถามเฉลี่ยอยู่ที่ 1.911 วินาที โครงงานนี้ยังสามารถอัพโหลดวิดีโอและเชื่อมต่อกับ API จาก Cutt.ly, IQAir และ
OpenWeatherMap ได้ โดยโครงงานนี้เป็นตัวอย่างที่ดีของการใช้ Dialogflow และ Firebase
ในการสร้างแชทบอทที่มีประสิทธิภาพและเป็นประโยชน์แก่ผู้ใช้งานที่เข้ามาใช้บริการ

\section{\ifenglish Challenges\else ปัญหาที่พบและแนวทางการแก้ไข\fi}

หลังจากทำการทดสอบโครงงานนี้ พบว่ามีปัญหาบางอย่างเกี่ยวกับการตอบคำถามที่ไม่ถูกต้อง
และความเร็วในการตอบคำถามที่อาจจะช้าไปบ้าง ดังนั้นเราสามารถแก้ไขปัญหาดังกล่าวได้โดยการ:
\begin{enumerate}
  \item ปรับปรุงโมเดลตัวช่วยเติมคำศัพท์ (autocomplete) ให้มีความแม่นยำและครอบคลุมคำศัพท์มากขึ้น เพื่อลดปัญหาการตอบคำถามที่ไม่ถูกต้อง
  \item ปรับแต่งโครงสร้างของระบบเพื่อเพิ่มประสิทธิภาพในการตอบคำถาม และลดความช้าในการตอบคำถาม
  \item พัฒนาระบบให้สามารถรับรู้ได้ว่าผู้ใช้งานได้เข้าใจคำตอบหรือยัง และจะสามารถแนะนำคำตอบที่เหมาะสมกับคำถามได้ในกรณีที่ตอบไม่ถูกต้อง
  \item พัฒนาระบบให้สามารถตรวจสอบความถูกต้องของคำตอบก่อนส่งคำตอบกลับไปยังผู้ใช้งาน และแสดงข้อความแจ้งเตือนในกรณีที่ตอบไม่ถูกต้อง
  \item ปรับปรุงเทคโนโลยีการแปลงคำพูดเป็นข้อความและการแปลงข้อความเป็นเสียงเพื่อเพิ่มความแม่นยำในการตอบคำถามและลดความล่าช้าในการตอบคำถาม
  \item ปรับปรุงการจัดการฐานข้อมูลในการเก็บคำถามและคำตอบ เพื่อให้สามารถค้นหาและดึงข้อมูล
\end{enumerate}


\section{\ifenglish%
Suggestions and further improvements
\else%
ข้อเสนอแนะและแนวทางการพัฒนาต่อ
\fi
}

ข้อเสนอแนะเพื่อพัฒนาโครงงานนี้ต่อไป มีดังนี้
\begin{enumerate}
  \item พัฒนาการตอบคำถามที่สามารถตอบได้มากกว่า 85\%: โดยการทำการเพิ่มข้อมูลให้กับ Dialogflow เพื่อเพิ่มความแม่นยำในการตอบคำถาม และสามารถใช้ Machine Learning ในการปรับปรุงความแม่นยำในการตอบคำถามได้เช่นกัน
  \item พัฒนาฟังก์ชันการสื่อสารที่ดีขึ้น: โดยการเพิ่มฟังก์ชันเสียงเพื่อให้ผู้ใช้งานสามารถใช้งานได้ง่ายขึ้น และเพิ่มความสามารถในการรับรู้เสียงและสื่อสารที่ดีขึ้น
  \item พัฒนาการเชื่อมต่อกับฐานข้อมูลที่ดีขึ้น: โดยการเพิ่มฟังก์ชันการเชื่อมต่อกับฐานข้อมูลแบบ NoSQL หรือ SQL เพื่อทำให้การจัดการข้อมูลง่ายขึ้นและทำให้ระบบทำงานได้รวดเร็วขึ้น
  \item พัฒนาการตรวจสอบความถูกต้องของข้อมูล: โดยการเพิ่มฟังก์ชันการตรวจสอบความถูกต้องของข้อมูลที่ผู้ใช้งานให้มา และแสดงข้อความแจ้งเตือนเมื่อมีข้อมูลที่ไม่ถูกต้อง
  \item พัฒนาระบบการทดสอบและการตรวจสอบความเสถียร: โดยการเพิ่มฟังก์ชันการทดสอบและการตรวจสอบความเสถียรในระบบ เพื่อให้ระบบทำงานได้ถูกต้องและไม่เกิดข้อผิดพลาดใดๆ ที่อาจจะเกิดขึ้นในอนาคต
\end{enumerate}