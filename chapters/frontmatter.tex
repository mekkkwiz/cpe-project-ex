\maketitle
\makesignature

\ifproject
\begin{abstractTH}
การเขียนรายงานเป็นส่วนหนึ่งของการทำโครงงานวิศวกรรมคอมพิวเตอร์
เพื่อโครงงานนี้มุ่งเน้นการพัฒนา chatbot ที่สามารถทำงานได้ผ่านเสียงของผู้ใช้งานเพื่อให้ความสะดวกสบายในการค้นหาเส้นทางภายในอาคาร
รวมถึงข้อมูลอื่น ๆ เช่น เวลา สภาพอากาศ และตำแหน่งของอาจารย์ โดยใช้เทคโนโลยี Text-to-Speech (TTS) ในการสื่อสารกับผู้ใช้งาน
วัตถุประสงค์ของโครงงานนี้คือ เพิ่มประสิทธิภาพการเข้าถึงข้อมูล ให้ข้อมูลเพิ่มเติมเช่นเวลา, สภาพอากาศ, และตำแหน่งของอาจารย์ ช่วยผู้ใช้งานในการค้นหาเส้นทางภายในอาคาร
และประยุกต์ใช้ Text-to-Speech (TTS) เพื่อสร้างเสียงสังเคราะห์ที่สื่อสารกับผู้ใช้งาน โครงงานนี้จัดทำขึ้นโดยนายวิศรุต ติ๊บบุ่ง และมีขอบเขตที่ ทำที่ตึก 30 ปี
ในมหาวิทยาลัยเชียงใหม่ ต้องเชื่อมต่อ Internet และรองรับการใช้งานผ่าน browser เท่านั้น ต้องใช้คอมพิวเตอร์ในการใช้งานเว็บไซต์
และรองรับการสนทนาและตอบคำถามภาษาไทยเท่านั้น โครงการนี้คาดว่าจะช่วยให้บริการด้านเส้นทางภายในอาคารและลดความเสี่ยงต่อการแพร่ระบาดของเชื้อโควิด-19
โดยลดการใช้งานมนุษย์และการใกล้ชิดกับผู้อื่น เมื่อโครงการสำเร็จ จะสามารถให้ประโยชน์ในด้านการสนับสนุนความสะดวกสบายในการเข้าถึงข้อมูล ลดการใช้ทรัพยากรมนุษย์
และป้องกันการแพร่ระบาดของเชื้อโควิด-19 โดยสามารถนำไปติดตั้งที่อาคารต่าง ๆ ได้ในอนาคต
\end{abstractTH}

\begin{abstract}
  Writing a report is an integral part of working on a computer engineering project.
  This project focuses on the development of a chatbot that can operate through user's voice input,
  providing convenience in searching for routes within buildings, as well as other information such as time,
  weather conditions, and the location of professors. This is achieved using Text-to-Speech (TTS) technology
  to communicate with users. The objective of this project is to increase the efficiency of accessing
  information, provide additional details such as time, weather, and the location of professors, assist
  users in finding routes within buildings, and apply Text-to-Speech (TTS) technology to create synthesized
  voices that communicate with users. This project is carried out by Mr. Wisarut Tibbung and is limited to
  the 30-year-old building at Chiang Mai University. It requires an internet connection and supports usage
  via a browser only, using a computer for website access, and supports Thai language conversation and question
  answering only. This project is expected to help provide services related to routes within buildings while
  reducing the risk of spreading COVID-19 by minimizing human interaction and close contact with others.
  Once completed, the project will benefit users by supporting their convenience in accessing information,
  reducing human resource usage, and preventing the spread of COVID-19, with the potential to be installed in
  various buildings in the future.
\end{abstract}

\iffalse
\begin{dedication}
This document is dedicated to all Chiang Mai University students.

Dedication page is optional.
\end{dedication}
\fi % \iffalse

\begin{acknowledgments}
ในโครงงานฉบับนี้ สำเร็จลุล่วงได้อย่างสมบูรณ์ด้วยความกรุณาอย่างยิ่งจาก
ผศ.ดร.ภาสกร แช่มประเสริฐ ที่ได้สละเวลาอันมีค่าแก่ผู้จัดทำโดยให้คำปรึกษาและแนะนำตลอดจน
ตรวจทานแก้ไขข้อบกพร่องต่างๆด้วยความเอาใจใส่เป็นอย่างยิ่ง จนโครงงานฉบับนี้สำเร็จสมบูรณ์
ลุล่วงได้ด้วยดีผู้จัดทำขอกราบขอบพระคุณเป็นอย่างสูงไว้ ณ ที่นี้ จากใจจริง
ขอขอบคุณ อ.ดร.ชินวัตร อิศราดิสัยกุล ที่ได้กรุณาให้คำแนะนำและให้ข้อมูลสำหรับการทำโครงงาน
ขอขอบคุณ รศ.ดร.อัญญา อาภาวัชรุตม์ วีระประพันธ์ ที่ได้กรุณาให้คำแนะนำและแนวมางในการปรับปรุงโครงงานชิ้นนี้
ขอขอบคุณ ผศ.ดร.ยุทธพงษ์ สมจิต ที่ได้กรุณาให้คำแนะนำและแนวมางในการปรับปรุงโครงงานชิ้นนี้

\acksign{2023}{3}{31}
\end{acknowledgments}%
\fi % \ifproject

\contentspage

\ifproject
\figurelistpage

\tablelistpage
\fi % \ifproject

% \abbrlist % this page is optional

% \symlist % this page is optional

% \preface % this section is optional
